\section{Conclusion}
\label{sec:conclusion}

We implemented an Android application as an interface that enables users to 
access to different image recognition APIs. In this way, we would like to 
propose a new way to speed up creation of training  datasets for machine 
learning algorithms, called \textit{crowd learning}. We believe that our 
application overcome the issue addressed to ImageNet in 
Section~\ref{sec:related}, since everyone can use an application on its 
smartphone - no usability problems - and it is possible to get rid of external 
image recognition services after getting over a critical threshold (solving 
\textit{zero-day learning} problem).
We tested different APIs, inferring that the best one is Computer Vision API by 
Microsoft Azure\cite{Microsoft}, followed by Imagga\cite{Imagga} and 
Watson\cite{IBM}. As discussed in Section~\ref{sec:results}, Google Reverse 
Image Search\cite{Google} is not made to label images, so we expected very low 
results. On the other hand, we evaluated the performance of OCR using a very simple 
benchmark, due to a lack of time. However, the results are really promising 
since it can recognize almost every printed characters.

We strongly hope that firms consider our proposal to improve machine learning 
algorithms and datasets, providing a free service to users all over the world 
and gaining free feedbacks on image classifications.