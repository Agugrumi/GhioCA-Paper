\section{Introduction}
\label{sec:introduction}

Nowadays, almost everyone owns at least one smartphone. That kind of devices 
are becoming even more efficient, with better hardware and performace. One of 
the most interesting capabilities of smartphones is to take high quality 
pictures, and users truly appreciate it. As a matter of fact, on the Google Play 
Store there are plenty of applications dedicated to photography.
A lot of people take photos in order to share it on social networks, such as 
Facebook, Instagram, Twitter and so on. Often those photos are associated with 
a description and, generally, some hashtags that label them. The latter could 
be personal words that a person associate to a certain photo (e.g. feelings, 
person names, places) but they could also be used in order to describes the 
content of the photo.
All these pieces of information can be really useful to train algorithms and to 
create datasets used in order to recognize images: in fact, these data, which 
are freely available, are posted by a person that manually label and describe a 
specific photo.

Usually, image recognition services process an image returning a set of labels 
associated to that picture. Results can vary in precision depending on the photo 
quality, subject and also on the algorithm used. 
The result given back often is not perfect: indeed, at the current state of the 
art, image recognition algorithms cannot be as precise as a human to describe a 
picture.
One of the opportunities to take advantage of this context is to embed in an 
application the possibility of taking photos which are processed by some image 
recognition service. The result of this process can be displayed to the 
user, who can choose only the subset of tags that she thinks are truly 
correlated to her picture: in that way users can automatically associate some 
tags to their photos and share them on social networks. On the other hand, using 
a human to check the result, you could evaluate how much a specific service 
works good, training your own dataset or algorithm.

Following this idea, we designed and developed \textit{GHio-Ca} (\textit{Giving 
Hashtags In Order to Classify Automatically}), an Android application that 
allows people to take photos or choose pictures which are automatically 
processed by some image recognition service. In particular, we used the 
following services:
\begin{itemize}
  \item Computer Vision API by Microsoft Azure \cite{Microsoft};
  \item Visual Recognition by IBM Watson \cite{IBM};
  \item Google Reverse Image Search \cite{Google};
  \item OCR Space \cite{OCR};
  \item Imagga \cite{Imagga}.
\end{itemize}

While quality of service and usability are major issues to develop an 
application, we focused mainly on technical aspects, such as the ability to 
take photos correctly and save them, make requests to image recognition 
services and parse results in order to display them. We did not focus our 
efforts on robustness of the application: if a request can not be performed 
(e.g. caused by a network failure or a service fault), the application notifies 
to the user that some error occurred. However, GHio-Ca allows users to retry the 
image recognition process on the same picture.

This paper is organized as follows: Section~\ref{sec:background} describes how 
artificial intelligence can perform image processing based on pattern matching 
and neural networks, while Section~\ref{sec:related} provides an overview of 
the current state of the art of Computer Vision applications. 
Section~\ref{sec:ghioca} describes how the application is designed and 
implemented, while Section~\ref{sec:issues} lists the problems we faced while 
developing our application. Finally, in Section~\ref{sec:results} we compare 
different image recognition APIs and in Section~\ref{sec:future} we propose some 
future developments for our application. Finally, Section~\ref{sec:conclusion} 
provides conclusions about our work.