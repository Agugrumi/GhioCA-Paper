% As a general rule, do not put math, special symbols or citations
% in the abstract
\begin{abstract}
Artificial intelligence is becoming more and more powerful. Thanks to its 
development, computers are becoming able to recognise object of the real world 
using camera, people's voices and spoken text using pattern matching 
techniques and microphones. This emerging field of study push industry, vendors 
and researchers to find better algorithms for images and characters recognition. 
In this context, many efforts have been done by big firms, such as Amazon, 
Google, IBM and Microsoft. In this paper, we present the design and the 
analysis of an Android application (namely GHio-Ca) developed in order to 
exploit the possibilities offered by image recognition APIs.
GHio-Ca is thought as an automatic photo tagger that could be used for training 
datasets. It is presented to the user as a camera app to share pictures on 
social networks.
Finally, we will compare the different APIs features. %FIXME
\end{abstract}

% keywords
\begin{IEEEkeywords}
 computer vision, android, ocr, image recognition.
\end{IEEEkeywords}