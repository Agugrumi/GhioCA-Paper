\section{GHio-Ca}
In this section we will provide some salient features implemented by our 
application, mainly regarding the architectural and implementation choices that 
we made developing the code.

\subsection{Architecture}
GHio-Ca is principally composed of two layers: (i) the layer that have to manage
network connectivity, uploading photos to the server and making requests to 
different services in order to make the image recognition or the translation of 
certain pieces of text, and (ii) the camera layer, that manages the photo 
capturing process and saving. 

These two layers are maintained as loose coupled as possible in order to make 
easy to change services used without making important changes to the overall 
application or changing the process of photo taking, without modify the 
networking layer.

%TODO
%	-add a UML of the application architecture at really high level
%		   NETWORK
%		
%		|-----------|		|-----------|
%		| LISTENERS |------>|	PHOTO	|
%		| ASYNCTASK |<------|			|
%		|-----------|		|-----------|

\subsection{Application description}
GHio-Ca (Giving Hashtags In Order To Classify Automatically) is an Android 
application that gives to a user the possibility to make image recognition on 
her pictures. This application embed some camera features in order to make 
possible to take a photo and recognize it directly, but it also allows to pick 
a picture from the saved ones and launch the classifying process on it. It 
supports both the optimization to recognize objects and to recognize text: the 
limitation is that the user have to choose it before starting the recognition 
process from the hamburger menu placed in the main activity. 

The overall application is mainly composed by four activities:
\begin{itemize}
	\item the splash activity: this view is used to request to the user to give 
		  the necessary permissions (network access, camera access, access to 
		  the storage);
	\item the camera preview activity: this one displays the user a minimal 
	      camera interface, with the possibility to take photos, turn off the 
	      flash, switch to frontal camera (if present) and to access to the 
	      gallery in order to pick a file instead of taking a picture. In the 
	      top left corner is present the hamburger menu: it allows user to 
	      choose the size of picture taken, remind the user to turn on the Wi-Fi 
	      sensor on every access and change the optimization of image recognition;
	\item the results activities: these are two different activity depending on 
	      the type of image recognition chosen. If "reverse image search" is 
	      selected the result activity will present, if no errors occurs, a 
	      textual description of the picture, with a bunch of hashtags correlated
	      to it. If "character recognition" is chosen, the result activity will 
	      present the text recognized.
\end{itemize}
