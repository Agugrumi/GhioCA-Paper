\section{Background}
\label{sec:background}

Machine learning is the field of artificial intelligence responsible for learning from data without being explicitly programmed to do so. In this way, the results provided by the algorithm do not depend on how data is processed, but on data itselves. Obviously, it is not possible to define a single algorithm for all possible learning problems.

There are mainly two methods to train an algorithm on a dataset: \textit{supervised} and \textit{unsupervised} learning. In supervised learning, we provide a label to specify which value is associated to each entry in the dataset. Consequently, we are able to train the algorithm based on examples. In unsupervised learning, there is no label for each entry. Thus, to determinate if two examples are referring to the same result, we need a \textit{similarity measure} (e.g. if we represent data using vector, one possible measure of similarity is cross product).

\textit{Pattern Matching} is the branch of machine learning responsible for detecting pattern and regularities in data. Tipically, it is implemented through supervised learning: the dataset is composed by examples with an associated label. In this way, the algorithm learns which are the features of a specific pattern (described by the label).

One of the most used method in machine learning for image classification is \textit{deep learning}: it uses an Artificial Neural Network (or ANN), which is a Neural Network (NN) composed by more than one hidden layer, in order to classify an image, based on similarity measures (unsupervised) or training examples (supervised). Tipically, NNs use a \textit{backpropagation} algorithm, composed of two phases: \textit{feed forward} and \textit{backward propagation}. Firstly, the image is decomposed in a vector-like representation. Secondly, during feed forward phase, the vector is fed as input to the NN and it is computed. Finally, during back propagation, the result of feed forward phase is compared to the real label; in case of mismatch, the wrong parts are back propagated into the NN in order to compensate the wrong behaviour. This process is repeated for each entry in the dataset.
\todo{image of NN}
Deep Neural Networks (DNNs) can use the same principles of NNs, which implement backpropagation, but decompose data on many levels (thus "deep" network). This method is based on the assumption that data can be \textit{abstracted} or \textit{composed} in many levels, simplifing the task of comparing pieces of data.
Usually, for \textit{computer vision} tasks (e.g. \cite{Handwritten}) \textit{Convolutional Neural Networks} (CNNs) are used\cite{CNN}.
However, this algorithms do not have perfect performances: in some cases, the label assigned to an image may be wrong. As a tradeoff between reliability and usability, classifiers usually assign a \textit{probability} to a certian label. In this way, who uses the algorithm can decide a confidence interval to keep/discard classifications. 