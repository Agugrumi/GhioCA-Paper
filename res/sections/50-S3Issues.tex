\section{Known Application issues}
\label{sec:issues}

\subsection{Services issues}

During the app development, we encountered different problems with some services, 
that are described below.

\subsubsection{Azure Image Recognition}

Azure Image Recognition is the core of GHio-CA and it works pretty well. 
Unfortunately, we had some problems with the service provider: firstly, it 
banned our first account; secondly it did not accept our student subscription 
that would have enabled us to take advantage of it. Finally, without any 
notification, Microsoft restored our first account and we were able to use the 
Image Recognition service.

\subsubsection{Free OCR Space Character Recognition}

Free OCR Space is an OCR service that works very well, providing the text 
contained in a picture. It comes with different kind of subscriptions: the 
free one allowed us to upload images up to 1MB. We solved this limitation 
using different providers combination to be able to use Free OCR Space when 
the image is smaller than 1MB, and other services when it is bigger than this 
threshold.

\subsubsection{IBM Watson Image Recognition}

We also adopted an Image Recognition service from IBM to obtain more 
tags from a shot performed by the user. Although we did not have problem with 
the service per-se, we found a bug in the Watson SDK related to the data 
format used, that we where able to fix.

\subsection{Smartphones issues}

During GHio-CA development we found problems with a particular smartphone, 
that is described here.

\subsubsection{S3 first application run}

We found that using the Samsung S3 (I9300) model permissions were not granted correctly, consequently we were not able to save the user's photos properly and thus to perform reverse image search at first application run. Debugging did not produce any evidence of bugs in our application, since the permissions 
were granted during the start up time.
We address this issue to the custom ROM installed on the smartphone, LineageOS 
(Android Nougat 7.1), which probably has some problem to handle authorizations only on S3 smartphones. In fact, we tried this ROM on a Nexus 5 and we have not noticed any erroneous behavior. To confirm our guesses, we tried another camera application available on the Google's Play Store, OpenCamera\footnote{\url{https://play.google.com/store/apps/details?id=net.sourceforge.opencamera}}
, and a Google's app, Snapseed\footnote{\url{https://play.google.com/store/apps/details?id=com.niksoftware.snapseed}}, experiencing the same problem.