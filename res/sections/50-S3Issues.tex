\section{Know Application problems}

\subsection{Problems with Services}

We encountered different problems with some services, that are described below.

\subsubsection{Azure Image Recognition}

Azure Image Recognition is the main GhioCA backbone, and it performs image 
recognition with an image description. It's a very powerful service. 
Unfortunately we had problems with the service provider, that first of 
all banned us without any reason and secondly it didn't accept our student 
subscription that'd have allowed us to use it.
And the end, without notifications, our first account was unbanned and we where 
able to use the platform.

\subsubsection{Free OCR Space Character Recognition}

Free OCR Space it's an OCR\footnote{OCR stands for \textit{Optical Character 
Recognition}.} service that works very well. This service comes with different 
subscription, and our free subscription allowed us to upload images up to 1MB. 
We solved this limitation using different providers combination in order to be 
able to use Free OCR Space when the image is less than 1MB, and other services 
when is bigger than 1MB. % TODO what are the names of the other services...?

\subsubsection{IBM Watson Image Recognition}

We used image recognition from IBM in order to be able to gain more tags from 
a shot performed by the user. % TODO need some rewording
Although we did not have problem with the service per-se, we found a bug in the 
Android SDK that we where able to fix. % TODO write about the fix

\subsection{Problems with Smartphones}

During GhioCA development we found problems with particular smartphones, that
will be listed here.

\subsection{S3 first application run}

We found that with the Samsung S3 (I9300) model permissions were not granted
correctly, resulting in the impossibility to save the users' photos and thus to
perform reverse image search only at the first application run. Debugging did
not produce any evidence of bugs in our application, since the permissions were
granted during App initialization.
We address the issue to the custom ROM flashed on the smartphone, LineageOS
(Android 7.0 Nougat), that probably has some problem handling authorization on
S3 smartphones (since we tried this ROM on a Nexus 5 and we haven't noticed any
strange behavior).
To confirm our guesses we tried another camera available on the Google Play
Store, OpenCamera\footnote{For more infos:
\url{https://play.google.com/store/apps/details?id=net.sourceforge.opencamera}},
 an open-source camera that requires Android 4.0.3 (Ice Cream Sandwich) or
later.
Opening it for the first time, granting the permission to write in the external
storage and taking a photo hasn't worked even for this application, confirming
our guesses about a bug in the LineageOS S3 version.
We were not able to solve this issue without restarting the application.
