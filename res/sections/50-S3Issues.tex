\section{Known Application issues}
\label{sec:issues}

\subsection{Services issues}

We encountered different problems with some services, that are described below.

\subsubsection{Azure Image Recognition}

Azure Image Recognition is core of GHio-CA. It is a very advanced service.
Unfortunately, we had some problems with the service provider: firstly,
it banned our first account; secondly it did not accept our student
subscription that would have enabled us to use it.
Finally, without any notification, Microsoft restored our first account and we
where able to use the Image Recognition service.

\subsubsection{Free OCR Space Character Recognition}

Free OCR Space is an OCR service that works very well, providing the text contained in a 
picture.
It comes with different kind of subscriptions: the free one allowed us to upload
images up to 1MB. 
We solved this limitation using different providers combination in order to use
Free OCR Space when the image is smaller than 1MB, and other services 
when it is bigger than this threshold.

\subsubsection{IBM Watson Image Recognition}

We also used an Image Recognition service from IBM in order to obtain more tags
from a shot performed by the user.
Although we did not have problem with the service per-se, we found a bug in the 
Watson SDK related to the data format used, that we where able to fix.

\subsection{Smartphones issues}

During GhioCA development we found problems with particular smartphones, that
are listed here.

\subsection{S3 first application run}

We found that with the Samsung S3 (I9300) model, permissions were not granted
correctly, consequently we were not able to save the user photos propperly and 
thus to perform reverse image search at first application run. Debugging did
not produce any evidence of bugs in our application, since the permissions were
granted on start up time.
We address this issue to the custom ROM flashed on the smartphone, LineageOS
(Android Nougnat 7.1), which probably has some problem to handle authorizations
on S3 smartphones. We tried this ROM on a Nexus 5 and we haven't noticed any
strange behavior.
To confirm our guesses, we tried another camera available on the Google's Play
Store, OpenCamera\footnote{For more infos:
\url{https://play.google.com/store/apps/details?id=net.sourceforge.opencamera}},
 an open-source camera that requires Android 4.0.3 (Ice Cream Sandwich) or
later.
Opening it for the first time, granting the permission to write in the external
storage and taking a photo hasn't worked even for this application, confirming
our guesses about a bug in the LineageOS version for Samsung S3.
We were not able to solve this issue without a restart of the application.
We found this problem also with Google's app Snapseed.
However, we do not address this as a main issue since there are few old devices
on the market which use this ROM.
